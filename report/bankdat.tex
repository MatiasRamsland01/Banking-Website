\documentclass[11pt,a4]{report}
\usepackage[utf8]{inputenc}
\usepackage[T1]{fontenc}
\usepackage[norsk]{babel}
%\renewcommand{\rmdefault}{cmss}    %forandrer font til times roma
\usepackage{verbatim,amsmath}
\usepackage{enumerate}
\usepackage{fancyhdr,fancyvrb}
\usepackage{graphicx,color,boxedminipage}
\usepackage{ragged2e,colortbl,appendix}
\usepackage{here,hyperref,multirow,pdfpages}
\usepackage[font=small,labelfont=bf]{caption}

\usepackage{xcolor}
\usepackage{lipsum,todonotes} 

% viser til figurer ligger 
%\usepackage[../FigurerFraProsjektene/DiverseFigurer/, 
%../FigurerFraProsjektene/MineFunksjoner/,
%../FigurerFraProsjektene/Prosjekt01_NumeriskIntegrasjon/,
%../FigurerFraProsjektene/Prosjekt02_Filtrering/,
%../FigurerFraProsjektene/Prosjekt03_Derivasjon/,
%../FigurerFraProsjektene/Prosjekt04_ManuellKjoring/,
%../FigurerFraProsjektene/Prosjekt05_Navn/,
%../FigurerFraProsjektene/Prosjekt06_Navn/,
%../FigurerFraProsjektene/Prosjekt07_Navn/,
%../FigurerFraProsjektene/Prosjekt08_Navn/,
%../FigurerFraProsjektene/Prosjekt09_navn/,
%../FigurerFraProsjektene/Prosjekt10_Navn/,
%../RapportData/Funksjoner/,
%]{figurepath} 

% Her utvider du med nye mapper


% inkludering av kode 
\usepackage{listings}
%\usepackage[framed,numbered,autolinebreaks,useliterate]{mcode}
% for å kunne skrive norske kommentarer i Matlab og vises i listingspakken
%\lstset{inputencoding=ansinew}

\lstset{language=Matlab,%
    %basicstyle=\color{red},
    inputencoding=ansinew,
    breaklines=true,%
    morekeywords={matlab2tikz},
    keywordstyle=\color{blue},%
    morekeywords=[2]{1}, keywordstyle=[2]{\color{black}},
    identifierstyle=\color{black},%
    stringstyle=\color{mylilas},
    commentstyle=\color{mygreen},%
    showstringspaces=false,%without this there will be a symbol in the places where there is a space
    numbers=left,%
    numberstyle={\tiny \color{black}},% size of the numbers
    numbersep=9pt, % this defines how far the numbers are from the text
    emph=[1]{for,end,break},emphstyle=[1]\color{red}, %some words to emphasise
    %emph=[2]{word1,word2}, emphstyle=[2]{style},    
}

\definecolor{darkgreen}{rgb}{0.0, 0.6, 0.0}
\definecolor{darkred}{rgb}{0.6, 0.0, 0.0}
\definecolor{gray}{rgb}{0.6, 0.6, 0.6}
\definecolor{mygreen}{RGB}{28,172,0} % color values Red, Green, Blue
\definecolor{mylilas}{RGB}{170,55,241}


% kan definere bredere tekstbredde og -høyde 
\textwidth135mm
\textheight195mm
\parindent0mm  % ingen innrykk ved begynnelsen av avsnitt

\setlength{\marginparwidth}{3cm}

\DeclareUnicodeCharacter{2212}{-}

\begin{document}
\setlength{\parskip}{0.5cm}   % denne lager 5mm avstand ved avsnitt
\selectlanguage{norsk}

% topptekst
\pagestyle{fancyplain}
\renewcommand{\chaptermark}[1]{\markboth{#1}{#1}}
\renewcommand{\sectionmark}[1]{\markright{\thesection\ #1}}
\lhead[\fancyplain{}{\bfseries\thepage}]{\fancyplain{}{\bfseries\rightmark}}
\rhead{}
\chead{}
\cfoot{\bfseries\thepage}
\lfoot{}
\rfoot{}


\renewcommand{\lstlistingname}{Kode}% Listing -> Kode


% forsidetabell
\begin{table}[hb]
	\centering
              \begin{tabular}{|l|lll|}\hline
                \multicolumn{4}{|l|}{\hspace*{130mm}}\\
                \multicolumn{4}{|l|}{DAT250 TODO}\\[-7mm]
              %\multicolumn{4}{|r|}{\scalebox{0.4}{\includegraphics{uis_nor_black}}}\\[15mm]
                \multicolumn{4}{|c|}{\Huge \bf PROSJEKT - HØSTEN 2021
                }\\[5mm]\hline
                & & &  \\[-3mm]
                Prosjekt- & \multicolumn{3}{|l|}{Bank Dat250;} \\
                oppgaven & \multicolumn{3}{|l|}{}\\[2mm]\hline
                \multicolumn{4}{c}{}\\[5mm]\hline
                & & &  \\[-3mm]
                Gruppenavn? & \multicolumn{3}{|l|}{\color{red}{AlphaBank?}} \\[2mm]\hline
                & & &\\[-3mm]
                Gruppens  & Navn &  Studentnummer & \\
                medlemmer  &   &   &  \\[2mm]
                & --- & ---     &  \\[6mm]
                & Jakub Mroz  & 260703       & \\[6mm]
                & Lukasz Pietkiewicz & 253469 & \\[20mm]
                 \hline
              \end{tabular}
\end{table}


% ingen sidetall på forsiden
\thispagestyle{empty}

\newpage

% romerske tall før kap.1
\pagenumbering{roman} 
  
% Innholdsfortegnelse
% Første linje legger selve innholdsfortegnelsen inn i
% innholdsfortegnelsen. Dette må gjøres manuelt på de kapitlene
% uten nummer

\addcontentsline{toc}{chapter}{\protect\numberline{}Innhold} 
\tableofcontents

\newpage


%\input{sammendrag.tex}

% Siden sammendraget er uten kapittelnummer, legges dette manuelt inn
% i innholdsfortegnelsen
\addcontentsline{toc}{chapter}{\protect\numberline{}Sammendrag}

\newpage


% start vanlig nummering
% Side 1 skal ALLTID være der hvor kapittel 1 starter
\pagenumbering{arabic}

% rett høyre- og venstremarg
\justifying

% ingen innrykk ved nytt avsnitt
\setlength{\parindent}{0em} 

%\input{kap1_integrasjon.tex}
%\input{kap2_filtrering.tex}
%\input{kap3_derivasjon.tex}
%\input{kap4_manuell_kjoring.tex}
%\input{kap5_automatisk_kjoring.tex}
%\input{kap6_rekkefølge_int_dev.tex}
%\input{kap7_rekkefolge_int_dev.tex}
%\input{kap8_kreativ_komponist.tex}
%\input{kap9_automatisk_kjoring_pid.tex}
%\input{kap10_beregning_hastighet.tex}
%\input{konklusjon.tex}


% Første linje er bibliografistilen, her finner flere varianter som du
% prøve. Andre linje er selve filen med dine referanser. Siste linje
% legger bibliografi inn i innholdsfortegnelsen. 

\bibliographystyle{plain}
%\bibliography{referanser.bib}
\addcontentsline{toc}{chapter}{Bibliografi} 

\appendix

\addcontentsline{toc}{chapter}{Vedlegg}
\renewcommand{\appendixname}{Vedlegg}
\renewcommand{\appendixpagename}{Vedlegg}



\chapter{Timelister}\label{ch:vedlegg_A}


%\chapter{Programlisting egendefinerte funksjoner}\label{ch:vedlegg_B}
%\section{Derivation.m}
%\lstinputlisting{Derivation.m}
%\section{EulerForward.m}
%\lstinputlisting{EulerForward.m}
%\section{FIR\_filter.m}
%\lstinputlisting{FIR_filter.m}
%\section{IIR\_filter.m}
%\lstinputlisting{IIR_filter.m}
%\section{Trapes.m}
%\lstinputlisting{Trapes.m}

%\chapter{Programlisting prosjekt 01}\label{ch:vedlegg_C}
%\section{P01\_F4\_MathCalculations.m}
%\lstinputlisting{P01_F4_MathCalculations.m}

%\chapter{Programlisting prosjekt 02}\label{ch:vedlegg_D}
%\section{P02\_F4\_MathCalculations.m}
%\lstinputlisting{P02_F4_MathCalculations.m}

\end{document}


